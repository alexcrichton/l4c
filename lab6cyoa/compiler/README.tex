\documentclass{article}
% \usepackage{graphs}
\usepackage{fullpage}
\usepackage{array}
\usepackage{setspace}
\newcommand{\wild}{\mbox{\tt\char"5F}}

\newcommand{\nonterm}[1]{$\langle${#1}$\rangle$}
\newcommand{\tok}[1]{$\langle$\emph{#1}$\rangle$}
\newcommand{\term}[1]{\textbf {#1}}
\newcommand{\OR}{\ensuremath{\ | \ \ }}

\newcommand{\proves}{\vdash}

\newcommand{\G}{\Gamma}
\newcommand{\cons}[2]{#1, \, #2}
\newcommand{\typed}[2]{#1 : #2}
\newcommand{\valid}[1]{#1 \; \mathit{valid}}
\newcommand{\typof}[3]{{#1} \proves \typed{#2}{#3}}


\title{Assignment 6\\ Object-oriented C0}
\author{15-411: Compiler Design\\
Alex Crichton (acrichto) and Robbie McElrath (rmcelrat)}

\begin{document}
\maketitle
\doublespacing

\section{Introduction}

We decided to add object-oriented features to to C0 for our lab 6. We strove to
maintain compatibility as much as possible with \emph{L4}, while supporting the
major features of OO programming that one would expect when working with a
language like C++ or Java.

Our compiler supports the basics of object oriented programming: classes,
single-class inheritance, method overriding, constructors, protected/private
access to fields/methods, and \texttt{super} invocation of parent methods.

\section{Specification}

\subsection{Grammar}
%\begin{figure}
\renewcommand{\arraystretch}{1.4}
\begin{tabular}{lcl}
  \nonterm{gdecl}      & ::= & $\cdots$ \OR \nonterm{cdecl} \OR \nonterm{mdef} \\
  \nonterm{cdecl}      & ::= & \term{class} \term{ident} \nonterm{ext-opt} \term{\{} \nonterm{citem-list} \term{\}} \term{;} \\
  \nonterm{ext-opt}    & ::= & $\epsilon$ \OR \term{extends} \term{ident} \\
  \nonterm{citem-list} & ::= & $\epsilon$ \OR \nonterm{citem} \nonterm{citem-list} \\
  \nonterm{citem}      & ::= & \nonterm{field} \OR \nonterm{fdecl} \OR \term{public:} \OR \term{private:} \\
  \nonterm{mdef}       & ::= & \term{ident}::\term{ident} \nonterm{param-list} \nonterm{block} \\
                       & \OR & \nonterm{type} \term{ident}::\term{ident} \nonterm{param-list} \nonterm{block} \\
  \nonterm{exp}        & ::= & $\cdots$ \OR \term{new} \term{ident} \nonterm{arg-list} \\
                       & \OR & \term{super}  \nonterm{arg-list} \OR \term{super} \verb"->" \term{ident} \nonterm{arg-list} \\
\end{tabular}
%\label{fig:grammar}
%\end{figure}

\subsection{Static Semantics}

\subsection{Dynamic Semantics}


\section{Implementation}

\section{Testing}

\section{Analysis}

We included many features of object oriented languages, but we still lacked some
major features like static methods/fields, method overloading, implicit
constructors, <ADD MORE HERE>.

After adding object oriented features, we realized that our generated assembly
has a lot of room for improvement. Because our IR only minimally changed, all of
our previous optimizations still worked in place, but adding these new features
introduced a whole new class of generated assembly which could be much better
optimized. Optimizations like memory aliasing and address arithmetic are much
more relevant with this language extension than they were before.

\end{document}
